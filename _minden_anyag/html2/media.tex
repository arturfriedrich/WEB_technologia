\subsection{Média támogatás}

%124
\begin{frame}
  A HTML5-ben beépülő modulok (pl. Flash) és akár programozás nélkül 
  lehet videót lejátszani a \texttt{<video>} elemmel (de a böngészők codec 
  támogatása hiányos). Használható formátumok:
  \begin{itemize}
    \item MP4 (video/mp4, \hiv{\href{https://caniuse.com/\#feat=mpeg4}{legjobb böngésző támogatás}})
    \item WebM (video/webm)
    \item Ogg (video/ogg)
  \end{itemize}
  Ha biztosra akarunk menni: publikálás több formátumban is.\\
  Formátumok közötti konvertálás: pl. 
  \hiv{\href{http://www.mirovideoconverter.com/}{MiroVideoConverter}}\\
  A lejátszás programozható a \hiv{\href{https://developer.mozilla.org/en-US/docs/Web/API/HTMLMediaElement}{Media API}}-val\\
\end{frame}

%125
\begin{frame}
  Ha a \texttt{<video>} elem nem támogatott egy böngészőben, a 
  címkék közötti szöveg jelenik meg. Opcionális attribútumok:
  \begin{description}[m]
    \item[\texttt{autoplay}] \hfill \\ A lejátszás azonnal indul; 
    \kiemel{nem ajánlott}, zavarhatja a felhasználót
    \item[\texttt{controls}] \hfill \\ Vezérlő gombokat jelenít meg
    \item[\texttt{width}, \texttt{height}] \hfill \\ A lejátszó 
    ablak szélessége, magassága képpontokban; \kiemel{ajánlott} megadni
    \item[\texttt{loop}] \hfill \\ Végtelenített lejátszás
  \end{description}
\end{frame}

%126
\begin{frame}
  \begin{description}[m]
    \item[\texttt{muted}] \hfill \\ Némítás
    \item[\texttt{poster}] \hfill \\ Egy kép, amit a betöltés 
    alatt / lejátszás megkezdéséig lát a felhasználó. Érték: URL
    \item[\texttt{preload}] \hfill \\ Adatfolyam betöltési módja. 
    Érték: \texttt{auto | metadata | none}.
    \item[\texttt{src}] \hfill \\ Videó forrása. \kiemel{Nem 
    ajánlott} a használata, mert csak egyetlen forrás nevezhető meg, 
    amit valószínűleg nem támogat minden böngésző. Érték: URL.
  \end{description}
\end{frame}

%127
\begin{frame}
  \begin{exampleblock}{\textattachfile{video1.html}{video1.html}}
    \footnotesize
    \lstinputlisting[style=HTML,linerange={8-13},numbers=left,firstnumber=8]{video1.html}
  \end{exampleblock}
    \begin{center}
    \includegraphics[scale=.2]{video1.png}
  \end{center}
\end{frame}

%128
\begin{frame}
  Több adatforrás is megadható beágyazott \texttt{<source>} elemekkel, melyek 
  közül a böngésző az első támogatott formátumhoz tartozót 
  fogja választani. Attribútumok:
  \begin{description}[m]
    \item[\texttt{src}] \hfill \\ Adatforrás. Érték: URL
    \item[\texttt{type}] \hfill \\ A forrás MIME típusa.
  \end{description}
  A \texttt{<video>} elemmel történő kísérletezéshez egy 
  \hiv{\href{http://v4e.thewikies.com/}{érdekes eszköz}}.
\end{frame}

%129
\begin{frame}
  \begin{exampleblock}{\textattachfile{video2.html}{video2.html}}
    \scriptsize
    \lstinputlisting[style=HTML,linerange={8-22},numbers=left,firstnumber=8]{video2.html}
  \end{exampleblock}
\end{frame}

%130
\begin{frame}
  A videók feliratozhatóak is a \texttt{<track>} elemmel. Felirat 
  formátum: \hiv{\href{https://www.w3.org/TR/webvtt1/}{VTT}}. 
  \hiv{\href{https://www.nikse.dk/SubtitleEdit/Online}
  {Online szerkesztő}}, 
  \hiv{\href{https://subtitletools.com/convert-to-vtt-online}{átalakító}}. 
  Attribútumok:
  \begin{description}[m]
    \item[\texttt{default}] \hfill \\ Kijelölhető több feliratsáv 
    közül az alapértelmezett.
    \item[\texttt{kind}] \hfill \\ Feliratsáv típusa: 
    \texttt{captions | chapters | descriptions | metadata | 
    subtitles} (ez az alapértelmezés).
    \item[\texttt{label}] \hfill \\ Feliratsáv címkéje, pl. a 
    felirat nyelve.
    \item[\texttt{src}] \hfill \\ A felirat forrása, kötelező. 
    Érték: URL
    \item[\texttt{srclang}] \hfill \\ Felirat nyelvének ISO 639-1 
    kódja, pl. hu.
  \end{description}
\end{frame}

%131
\begin{frame}
  \begin{exampleblock}{\textattachfile{video3.html}{video3.html} 
  (\textattachfile{subtitle.vtt}{subtitle.vtt})}
    \scriptsize
    \lstinputlisting[style=HTML,linerange={8-23},numbers=left,firstnumber=8]{video3.html}
  \end{exampleblock}
\end{frame}

%132
\begin{frame}
  Feladat: készítse el a Big Buck Bunny webes lejátszóját!
  \begin{columns}[c]
    \column{0.45\textwidth}
      \begin{itemize}
        \scriptsize
        \item A videó felbontása 800x450 képpont.
        \item Poszter fotó: 
        \texttt{https://upload.wikimedia.org/wikipedia/}
        \texttt{commons/thumb/c/c5/}
        \texttt{Big\_buck\_bunny\_poster\_big.jpg/}
        \texttt{800px-Big\_buck\_bunny\_poster\_big.jpg}
        \item Két adatforrás is van, ezeket kell felajánlani: 
        \texttt{https://download.blender.org/peach/}
        \texttt{bigbuckbunny\_movies/}
        \texttt{big\_buck\_bunny\_480p\_stereo.ogg}
        és ugyanezen az útvonalon \texttt{big\_buck\_bunny\_480p\_h264.mov} 
        néven.
        \item Feliratsáv nem áll rendelkezésre.
        \item Biztosítsa a letöltés lehetőségét, ha a böngésző nem 
        támogatja a lejátszást!
      \end{itemize}      
    \column{0.45\textwidth}
      \begin{exampleblock}{\textattachfile{video4.html}{video4.html}}
        \begin{center}
          \includegraphics[width=\textwidth]{video4.png}\\
        \end{center}
      \end{exampleblock}
  \end{columns} 
\end{frame}

%133
\begin{frame}
  Az \texttt{<audio>} elemmel lehet hangokat, zenét lejátszani. 
  Szabványos formátumok:
  \begin{itemize}
    \item MP3 (audio/mpeg, \hiv{\href{https://caniuse.com/\#feat=mp3}
    {legjobb böngésző támogatás}})
    \item Wav (audio/wav)
    \item Ogg (audio/ogg)
  \end{itemize}
  Attribútumai a méretektől eltekintve azonosak a 
  \texttt{<video>}-éival. \\
  Itt is célszerű az adatforrást beágyazott \texttt{<source>} 
  elemekkel megadni, és a lejátszást nem támogató böngészőknél 
  hibaüzenetet megjeleníteni.
\end{frame}

%134
\begin{frame}
  \begin{exampleblock}{\textattachfile{audio1.html}{audio1.html}
    (\textattachfile{dog.ogg}{dog.ogg}, 
     \textattachfile{dog.mp3}{dog.mp3})}
    \small
    \lstinputlisting[style=HTML,linerange={8-15},numbers=left,firstnumber=8]{audio1.html}
  \end{exampleblock}
  \begin{center}
    \includegraphics[scale=.5]{audio1.png}\\
  \end{center}
\end{frame}
