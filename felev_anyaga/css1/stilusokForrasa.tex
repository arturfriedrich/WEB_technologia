\subsection{Stílusok forrása}

%12
\begin{frame}
  Háromféle helyen lehet stílusokat megadni:
  \begin{enumerate}
    \item Külső fájlban (\texttt{css} kiterjesztés, \texttt{<link>} elem)
    \item A \texttt{<head>} elembe ágyazott \texttt{<style>} elemben. Csak akkor ajánlott, ha egyetlen HTML fájlt kívánunk formázni ezekkel a stílusokkal.
    \item Soron belül: a HTML elemek \texttt{style} attribútumának értékeként. Ismét \kiemel{keveredik a tartalom a stílussal}, ezért általában \kiemel{nem ajánlott} a használata!
  \end{enumerate}
\end{frame}

%13
\begin{frame}
  \begin{exampleblock}{\textattachfile{egyszeruSzelektor2.html}{egyszeruSzelektor2.html}}
    \footnotesize
    \lstinputlisting[style=HTML,linerange={3-12},numbers=left,firstnumber=3]{egyszeruSzelektor2.html}
    \lstinputlisting[style=HTML,linerange={16-16},numbers=left,firstnumber=16]{egyszeruSzelektor2.html}
  \end{exampleblock}
\end{frame}
