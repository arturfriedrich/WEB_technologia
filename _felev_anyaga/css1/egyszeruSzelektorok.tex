\subsection{Egyszerű szelektorok}

%7
\begin{frame}
  \begin{description}[m]
    \item[HTML elem neve] \hfill \\ \texttt{p \{ font-style: italic; \}}
    \item[Egyedi azonosító (\texttt{id} attribútum) alapján] \hfill \\ 
      \texttt{\#lablec \{ font-size: 10pt; \}}\\
      Az \texttt{id} nem kezdődhet számjegy karakterrel!
    \item[Univerzális szelektor, mindenre illeszkedik] \hfill \\ \texttt{* \{ font-size: smaller; \}}
  \end{description}
\end{frame}

%8
\begin{frame}
  \begin{description}[m]
    \item[Osztály (\texttt{class} attribútum alapján)] \hfill \\ 
      \texttt{*.kisbetus \{ font-size: small; \} /* bármilyen HTML elemhez */} \\
      \texttt{.kisbetus \{ font-size: small; \} /* bármilyen HTML elemhez, rövid alak */}\\
      \texttt{p.voros \{ color: red; \} /* csak adott (pl. <p>) HTML elemhez */}\\
      A \texttt{class} értéke nem kezdődhet számjeggyel, de lehet egyszerre több, szóközzel elválasztott értéke: \\
      \texttt{<p class="kisbetus voros">Apróbetűs piros bekezdés</p>}
    \item[Elemek csoportosítása] \hfill \\ \texttt{h1, h2, h3 \{ font-family: Arial; \}}
  \end{description}
\end{frame}

%9
\begin{frame}
  \begin{exampleblock}{\textattachfile{egyszeruSzelektor1.html}{egyszeruSzelektor1.html}}
    \scriptsize
    \lstinputlisting[style=HTML,linerange={3-13},numbers=left,firstnumber=3]{egyszeruSzelektor1.html}
  \end{exampleblock}
\end{frame}

%10
\begin{frame}
  \begin{exampleblock}{\textattachfile{egyszeruSzelektor1.html}{egyszeruSzelektor1.html}}
    \scriptsize
    \lstinputlisting[style=HTML,linerange={14-15},numbers=left,firstnumber=14]{egyszeruSzelektor1.html}
  \end{exampleblock}
\end{frame}

%11
\begin{frame}
  \begin{exampleblock}{\textattachfile{egyszeruSzelektor1.css}{egyszeruSzelektor1.css}}
    \scriptsize
    \lstinputlisting[style=HTML,numbers=left]{egyszeruSzelektor1.css}
  \end{exampleblock}
  \begin{center}
    \includegraphics[width=.9\textwidth]{egyszeruSzelektor1.png}
  \end{center}
\end{frame}
